%% LyX 2.2.2 created this file.  For more info, see http://www.lyx.org/.
%% Do not edit unless you really know what you are doing.
\documentclass[12pt]{article}
\usepackage[latin9]{inputenc}
\usepackage{geometry}
\geometry{verbose}
\setcounter{secnumdepth}{0}
\usepackage{polynom}
\include{longdiv}
\usepackage{textcomp}
\usepackage{mathtools}
%\usepackage{amsmath}
\usepackage{amssymb}
%\usepackage[unicode=true,
% bookmarks=false,
% breaklinks=false,pdfborder={0 0 1},backref=section,colorlinks=false]
% {hyperref}
%\usepackage{breakurl}
\usepackage{apacite}

\makeatletter

%%%%%%%%%%%%%%%%%%%%%%%%%%%%%% LyX specific LaTeX commands.
%% Because html converters don't know tabularnewline
\providecommand{\tabularnewline}{\\}

%%%%%%%%%%%%%%%%%%%%%%%%%%%%%% User specified LaTeX commands.


%\usepackage{exsheets}
\usepackage[toc,page]{appendix}
%\usepackage{amsthm} 
\usepackage{array}
\mathtoolsset{showonlyrefs} 
%\usepackage{graphicx}
%\usepackage{listings}
%\usepackage{indentfirst}
%\usepackage{setspace}
%\doublespacing
\usepackage{enumerate}
\usepackage{multicol}
\usepackage{commath}% Used for \abs{}
\usepackage{dcolumn}
\newcolumntype{2}{D{.}{}{2.0}}


\usepackage{pgf}
%\usepackage{tikz}
%\usepackage{mathrsfs}
%\usetikzlibrary{arrows}


\newcommand{\ssol}{\vspace{3em}}
\newcommand{\lsol}{\vspace{10em}}
\newcommand{\blnk}{{\underline {\hspace{1.5in}}}}
\newcommand{\Lim}[1]{\raisebox{0.5ex}{\scalebox{0.8}{$\displaystyle \lim_{#1}\;$}}}



\makeatother

\begin{document}

\begin{titlepage}
    \begin{center}
        \vspace*{1cm}
        
        \textbf{A study of the Fundamental Theorem of Algebra}
        
        \vspace{0.5cm}
        and its parallels to the integers.
        
        \vspace{1.5cm}
        
        \textbf{Justin Boyer}
        
        \vfill
        
        A thesis presented for the degree of\\
        Masters of Mathematics
        
        \vspace{0.8cm}
        
        \includegraphics[width=0.4\textwidth]{university.png}
        
        Department of Mathematics\\
        University of Utah\\
        United States\\
        June 02, 2017
        
    \end{center}
\end{titlepage}


\section{Introduction}

In mathematics, symmetry is often analyzed, utilized, or identified.
 The early
part of this paper sets out to help the reader identify the mathematical
analogy between the system of integers and the system of polynomials
with real coefficients. We will then see parallels between prime numbers
and irreducible polynomials, 'clock arithmetic' built from integers
using prime numbers and 'clock arithmetic' built from polynomials
using irreducible polynomials. Having explored this analogy in some
detail, we will turn our attention to the Fundamental Theorem of Algebra,
which states that any polynomial with complex coefficients has at
least one complex root. Our main goal, the one that will occupy the
remainder of this paper, is to use the understanding of the analogy
explored in the earlier sections to give a complete and accessible
exposition of an algebraic proof of the Fundamental Theorem of Algebra.
This paper seeks to expose at an
undergraduate college level one of the beautiful mathematical analogies
and to show how it is used to establish one of the most important
results in algebra.

\section{Comparing the integers and the polynomials with coefficients in the
rational, real or complex number fields}

Suppose we want to solve a polynomial equation 
\[
\frac{a_{d}}{b_{d}}x^{d}+\frac{a_{d-1}}{b_{d-1}}x^{d-1}+\ldots+\frac{a_{0}}{b_{0}}=0
\]
with rational numbers $\frac{a_{i}}{b_{i}}$ (where $a_{i},b_{i}$
are integers) as coefficients. We could turn it into a polynomial equation
with integer coefficients just by multiplying both sides of the above
equation by a common multiple of $\left\{ b_{d},b_{d-1},\ldots,b_{0}\right\} $.
So every root of a polynomial with rational coefficients is a root
of a polynomial with integer coefficients. A deeper fact, called Gauss's
Lemma, says that any factorization of a polynomial 
\[
a_{d}x^{d}+a_{d-1}x^{d-1}+\ldots+a_{0}
\]
with integer coefficients $a_{i}$ that cannot be factored into two
polynomial factors with integer coefficients cannot be factored into
two polynomial factors with rational coefficients. These two facts
led mathematicians to study systems of polynomials with coefficients
in a number system in which you could add, subtract, multiply $and$
divide. They called such a system a $field$. Besides the field of
rational numbers, denoted $\mathbb{Q}$, other examples of fields
are the field $\mathbb{R}$ of real numbers and the field $\mathbb{C}$
of complex numbers. They also noticed that the system $\mathbb{F}\left[x\right]$
of polynomials with coefficients in a field $\mathbb{F}$ behaves
a lot like the most familiar number system, the integers. Both have
unique factorization into prime factors, greatest common divisors,
least common multiples, etc.

\section{Parallels}
There are many parallels between the integers and the polynomials with integer coefficients.  For example, we can add using very similar algorithms:
\begin{table}[h!]
\begin{tabular}{cccc}
&1&2&3 \\
+&&4&5 \\ \hline
&1&6&8
\end{tabular}
\end{table}

Likewise if we add the polynomials $(x^2+2x+3)+(4x+5)$

\begin{table}[h!]
\begin{tabular}{cccc}
&$x^2+$&$2x+$&3 \\
+&&$4x+$&5 \\ \hline
& $x^2+ $& $6x+$&8
\end{tabular}
\end{table}

Similarly with the vertical multiplication algorithm allowing one modification we do not carry over the tens place when we multiply the polynomials:
\begin{table}[h!]
\begin{tabular}{cccc}
&1&2&3 \\
$\times$&&4&5 \\ \hline
 &6&1&5 \\
4&9&2&0 \\ \hline
5&5&3 &5
\end{tabular}
\end{table}

Likewise if we multiply the polynomials $(x^2+2x+3) \times (4x+5)$

\begin{table}[h!]
\begin{tabular}{cccc}
&$x^2+$&$2x+$&3 \\
$\times$&&$4x+$&5 \\ \hline
&$5x^2+$&$10x+$&15 \\
$4x^3+$&$8x^2+$&$12x$ \\ \hline
$4x^3+$&$13x^2+$&$22x+$&15
\end{tabular}
\end{table}

Notice if we substitute in $10$ for $x$ we will arrive at the integer solution.  Lastly the typical division algorithm:

\begin{multicols}{2}
\longdiv{1234}{11}
\columnbreak
\polylongdiv{x^3+2x^2+3x+4}{1x+1}
\end{multicols}

Again the two algorithms are very similar.  These algorithms are not the only that have these similarities, for example multiplying using the box method also maintains these parallels.  The reason for these similarities reduces to the fact that the variables in the polynomial keep track of the place values in the corresponding integer.  Given this we can create a mapping and show that addition, multiplication and in some sense division are preserved.



\section{The Mapping $\Phi_{b}$}

What we want to do is create a function (also called a mapping) that
takes a polynomial with integer coefficients to the (corresponding)
number system formed by its coefficients. For example we could ask
that the mapping substitute an integer $b$ for every \char`\"{}$x$'',
thereby taking the given polynomial with integer coefficients to an
integer. As an example of this, if $b=10$, the mapping would send
$2x^{2}+3x+4\xrightarrow[\Phi_{10}]{}234$. It is common to use the
letter $\mathbb{Z}$ to denote the integer number system and $\mathbb{Z}\left[x\right]$
to denote the ring of polynomials with integer coefficients.

\subparagraph{Example}

We would then denote the function (mapping) described above as 
\[
\begin{array}{c}
\Phi_{10}:\mathbb{Z}[x]\rightarrow\mathbb{Z}\\
f(x)\mapsto f(10).
\end{array}
\]
In other words, if $f(x)=5x^{4}+4x^{3}+2x+1$,

\[
\Phi_{10}(f(x))=f(10)=5(10)^{4}+4(10)^{3}+2(10)+1=5421.
\]

More generally 
\[
\Phi_{b}(f(x))=f(b)
\]


\subsection*{Properties of $\Phi_{b}$ }

The following proofs use two polynomials $f_{1},f_{2}$, with coefficients
in any of the number systems $\mathbb{Z}$, $\mathbb{Q}$, $\mathbb{R}$
or $\mathbb{C}$. 
\begin{itemize}
\item $\Phi_{b}$ preserves addition 
\end{itemize}
\emph{Proof:} 
\begin{align*}
\Phi_{b}(f_{1}(x)+f_{2}(x)) & =f_{1}(b)+f_{2}(b)\\
 & =\Phi_{b}(f_{1}(x))+\Phi_{b}(f_{2}(x))
\end{align*}

\begin{itemize}
\item $\Phi_{b}$ preserves multiplication 
\end{itemize}
\emph{Proof:} 
\begin{align*}
\Phi_{b}(f_{1}(x)f_{2}(x)) & =f_{1}(b)f_{2}(b)\\
 & =\Phi_{b}(f_{1}(x))\Phi_{b}(f_{2}(x))
\end{align*}

\begin{itemize}
\item A polynomial maps to zero if and only if the polynomial is divisible
by $x-b$ 
\end{itemize}
\emph{Proof:}

Given $f(x)=a_{n}x^{n}+a_{n-1}x^{n-1}+\cdots+a_{0}$, and a number
$r$ such that 
\[
f(r)=0.
\]

By long division of polynomials 
\[
a_{n}x^{n}+\cdots+a_{0}=(x-r)(c_{n-1}x^{n-1}+\cdots c_{0})+\text{constant}
\]
Substituting $x=r$ into the both sides of the equation above, 
\[
0=0+\text{constant}.
\]
So by necessity $\text{constant}=0$, therefore 
\[
f(x)=(x-r)(c_{n-1}x^{n-1}+\cdots c_{0}).
\]


\subsection*{A rough comparison of prime factorization}

This section will compare prime factorization in the ring of integers
and the ring of polynomials with coefficients in a field. Prime factorization
is decomposing something into its constituent primes. Among the integers
we have the \emph{the fundamental theorem of arithmetic}, which says
that every positive integer has a unique prime factorization. In math
speak this states that any positive integer $k\geq2$, $k$ may be
rewritten as 
\[
k=p_{1}^{l_{1}}p_{2}^{l_{2}}\cdots p_{n}^{l_{n}}
\]
where the $p_{i}$'s are the $n$ distinct prime factors, each of
order $l_{i}$. \\

Do the polynomials with coefficients in a field $\mathbb{F}$ have
a similar analogue? In this set-up, polynomials of degree zero, that
is, polynomials with only constant terms $a_{0}$, play the role of
$\text{\textpm1}$ in the integers, so that prime factors are always
polynomials of degree $d\geq1$. In grade school, polynomials are
often factored by breaking up the polynomial into several terms, each
of which can't be broken up further. Similarly these factored terms
may be multiplied together to retrieve the original polynomial.

For example in $\mathbb{Q}\left[x\right]$ 
\[
x^{2}-1=\left(x+1\right)\left(x-1\right)
\]
but 
\[
x^{2}-2
\]
is \char`\"{}prime'' in $\mathbb{Q}\left[x\right]$ since $\sqrt{2}$
is not in the number system $\mathbb{Q}$. Since $x^{2}-2$ is a polynomial
not an integer instead of prime it is more common to say irreducible.
In $\mathbb{R}\left[x\right]$, $x^{2}-2$ is reducible since $\sqrt{2}$
is an element of the reals, $\mathbb{R}$, i.e., 
\[
x^{2}-2=\left(x+\sqrt{2}\right)\left(x-\sqrt{2}\right).
\]
Similarly the prime factorization of $f(x)=x^{2}+1$ in $\mathbb{R}\left[x\right]$
is $x^{2}+1$. Since the square of any real number is greater or equal
to zero thus,
\[
x^{2}+1
\]
is irreducible in $\mathbb{R}\left[x\right]$. However in the system
of polynomials $\mathbb{C}\left[x\right]$, $x^{2}-1$ factors as
\[
x^{2}+1=\left(x+i\right)\cdot\left(x-i\right).
\]

In general, mathematicians decided that a polynomial $f\left(x\right)$
in $\mathbb{F}\left[x\right]$ is called irreducible if the polynomial
cannot be written as a product 
\[
f\left(x\right)=g\left(x\right)\cdot h\left(x\right)
\]
where $g\left(x\right)$ and $h\left(x\right)$ are polynomials in
the same system $\mathbb{F}\left[x\right]$ and the degrees of $g\left(x\right)$
and $h\left(x\right)$ are both less than the degree of $f\left(x\right)$.
In other words, $f\left(x\right)$ is irreducible if division by any
polynomial $g\left(x\right)$ in the same system $\mathbb{F}\left[x\right]$
always leaves a remainder. So the only polynomials that divide an
irreducible polynomial are the irreducible polynomial itself and the
constant polynomials $a_{0}$. Remind you of anything?

\subparagraph{Exercise:}

If you don't know already, try to figure out how to do prime factorization
in any system $\mathbb{F}\left[x\right]$ of polynomials where $\mathbb{F}$
is one of our fields. \\

The 'miracle' is that every polynomial $f(x)$ in $\mathbb{R}\left[x\right]$
can be factored into irreducible factors of degrees $1$ and $2$.
But then each irreducible factor 
\[
ax^{2}+bx+c
\]
of $f(x)$ with $a$, $b$ and $c$ real can be considered as a polynomial
in $\mathbb{C}\left[x\right]$. (Remember that real numbers are also
complex numbers, it's just that their imaginary part is zero). It
is then possible to factor $ax^{2}+bx+c$ in $\mathbb{C}\left[x\right]$
by the quadratic formula, 
\[
a\left(x-\frac{-b+\sqrt{b^{2}-4ac}}{2a}\right)\left(x-\frac{-b-\sqrt{b^{2}-4ac}}{2a}\right).
\]
In fact, any polynomial $f(x)$ in $\mathbb{R}\left[x\right]$ be
factored completely in $\mathbb{C}\left[x\right]$ as 
\[
f(x)=a_{d}\cdot(x-r_{1})^{l_{1}}(x-r_{2})^{l_{2}}\cdots(x-r_{i})^{l_{i}},
\]
the same is true for any $f(x)$ in $\mathbb{C}\left[x\right]$. We
know this fact as the \emph{Fundamental Theorem of Algebra}. The Fundamental
Theorem of Algebra states that the only irreducible polynomials in
$\mathbb{C}\left[x\right]$ are the polynomials of degree one! So,
roughly speaking, every $d$-th degree polynomial in $\mathbb{C}\left[x\right]$
has $d$ complex roots. These facts considered together with long
division of polynomials enable use to rewrite the polynomial 
\[
f(x)=a_{d}x^{d}+a_{d-1}x^{d-1}+\ldots+a_{0}
\]
as a product 
\[
f\left(x\right)=a_{d}\left(x-s_{1}\right)\cdot\cdots\cdot\left(x-s_{d}\right).
\]
I.e., if the $r_{i}$'s are the roots of $f(x)$ and the $l_{i}$
are the multiplicities of the roots, then each root $r_{i}$ occurs
in the list 
\[
s_{1},s_{2},\ldots,s_{d}
\]
exactly $l_{i}$ times.

\subparagraph{Exercise:}

For some small values of $d$, multiply out the right-hand-side of
the equality 
\[
a_{d}x^{d}+a_{d-1}x^{d-1}+\ldots+a_{0}=a_{d}\left(x-s_{1}\right)\cdot\cdots\cdot\left(x-s_{d}\right)
\]
so that you can give a formula for each quantity $\frac{a_{i}}{a_{d}}$
as a function of the quantities $s_{1},s_{2},\ldots,s_{d}$. Can you
guess the general formula for all values of $d$? Those formulas are
called the elementary symmetric functions of $s_{1},s_{2},\ldots,s_{d}$.
They will figure in an important way later on in this story.

\section{Toward a proof of the Fundamental Theorem of Algebra}

The Fundamental Thereon of Algebra is critical to much in algebra,
and the proof is often waved away as being beyond the scope of college
mathematics courses. In order to motivate the proof we start with
an example that highlights necessary components of the proof, while
giving some concreteness.

\section*{How many roots does $x^{3}-r_{1}x^{2}+r_{2}x-r_{3}$ have?}

Consider the following polynomial 
\[
p(x)=x^{3}-r_{1}x^{2}+r_{2}x-r_{3}
\]
how many roots does it have? Let's suppose that $r_{1},r_{2},r_{3}$
are real numbers and prove that $p(x)$ has three
roots in the complex field. We will take as a given that we know that,
as $x$ goes to $+\infty$, $p(x)$ becomes positive and, as $x$
goes to $-\infty$, $p(x)$ becomes negative. A slightly more complicated
fact is the fact that $p(x)$ is a continuous function of $x$, which
is often informally explained as the fact that you can draw the graph
of 
\[
y=p\left(x\right)
\]
without lifting your pencil from the page. So as a consequence of
the fact that $p(x)$ is continuous, your pencil cannot go from negative
$y$ to positive $y$ without crossing a place where $y=p\left(x\right)=0$.
That is, there is a real number $x$ where $p\left(x\right)=0$. So
\[
x^{3}-r_{1}x^{2}+r_{2}x-r_{3}
\]
has a real root $s_{1}$. As we have shown earlier, this means that
\[
x^{3}-r_{1}x^{2}+r_{2}x-r_{3}=\left(x-a_{1}\right)\left(x^{2}+bx+c\right).
\]
But now, again as had already been shown, this means that $x^{2}+bx+c$
can be factored into linear factors in $\mathbb{C}\left[x\right]$
using the quadratic formula.

Suppose now that we don't know that the coefficients of $x^{3}-r_{1}x^{2}+r_{2}x-r_{3}$
are real but just lie in some field $\mathbb{F}$, so the operations
on the coefficients of addition, subtraction, multiplication, and division hold. And suppose
we know that its roots lie in that same field. Let's call these roots
$a_{1},a_{2},a_{3}$.

Typically one would rewrite 
\[
x^{3}-r_{1}x^{2}+r_{2}x-r_{3}
\]
as 
\[
(x-a_{1})(x-a_{2})(x-a_{3}).
\]

Why can this polynomial be written as a product of linear factors? Recall the previous
section on the relationship between prime numbers and irreducible
factors of polynomials. It turns out that it is always possible to
factor a polynomial in $\mathcal{\mathbb{F}}\left[x\right]$ if $\mathbb{F}$
is a large enough field.

\subsection*{Field Extension}

The idea of a field extension boils down to increasing the set of
numbers one is able to use, while maintaining the properties of addition,
subtraction and multiplication and division. For example the Pythagoreans
worked in the field of rational numbers, therefore when they came
across $\sqrt{2}$ they did not believe it could be a solution. We
can now extend our field to all real values and now $\sqrt{2}$ is
a perfectly fine solution.

\subsubsection*{5-hour Clock Arithmetic}

The basics of field extension can be most easily understood by seeing
first how the integers can be turned into a field through the world
of clock (modular) arithmetic. Consider an algebraic structure that
consists of five elements, $\{0,1,2,3,4\}$. We will call this a 5-clock
arithmetic, because its arithmetic is just like what we use for the
hour hand on the clock, except that there are only five hours in our
'day,' that is, the set of hours has only five elements. It is possible
to add in this structure, for example, $1+1=2$, $1+2=3$, and $0+n=n$
thus 0 is the additive identity. But what about $4+1=$? On the 5-hour
clock we are forced to make $4+1=0$ since the hour after four o'clock
is the place where the hour-hand starts over. In other words in this
field the number 5 is mapped to the number 0, this implies the number
6 would be mapped to 1, and $7\rightarrow2$, $8\rightarrow3$, $9\rightarrow4,\ldots$.
Using these mappings and induction we can add any values in this structure
and still have values in this structure. For example addition in 5-clock
arithmetic looks like: 
\[
2+4=1
\]
\[
3+4=2
\]
\[
1+2+3+4=0
\]
\[
3+3+3+3+4=1
\]
so on and so forth. The last equation in the series highlights repeated
addition, we can use this model of multiplication (repeated addition)
to understand how multiplication functions in 5-clock arithmetic.
\begin{align*}
3+3+3+3+4 & =1\\
4\cdot3+4 & =1\\
4\cdot(3+1) & =1\text{ using the distributive property}\\
4\cdot4 & =1
\end{align*}
Thus the multiplicative inverse of 4 is 4.  Further if $4\cdot4=1$, then multiplying both sides by 4, we get 
\begin{align*}
\underbrace{4\cdot4}_{=1}\cdot4 & =4\\
1\cdot4 & =4
\end{align*}
There is a multiplicative identity in this structure, it is the number
1.

Let's take a closer look at multiplication of twos. 
\begin{align*}
2\cdot0=0\\
2\cdot1=2\\
2\cdot2=4\\
2\cdot3=1\\
2\cdot4=3\\
\end{align*}

We have a multiplicative inverse for 2 it is 3. I.e., $2\cdot3=1$,
therefore $2^{-1}=3$. Likewise there is a multiplicative inverse
for $1,3,4$ namely $1^{-1}=1$, $3^{-1}=2$, $4^{-1}=4$. We will
use the multiplicative inverse to understand division in this system.\\

What does $2\div4=?$ in 5-clock arithmetic? Rephrasing as a multiplication
problem, $2=?\cdot4$. We know 4 has a multiplicative inverse namely
itself, so we can multiply the previous equation by 4, 
\[
2\cdot4=?\cdot4\cdot4
\]
$2\cdot4=3$ and $4\cdot4=1$ in this clock 5 arithmetic, therefore
$3=?$. So 
\[
2\div4=3
\]
We can divide, because we have a multiplicative inverse.

A more efficient method for computing division exists. In fact it
was the Egyptians who were one of the first peoples to record this
more efficient method of division. Similar to the above process they
framed it as a multiplication problem with the multiplier (or multiplicand)\footnote{Exercise: Why does it not matter if it is the multiplier or multiplicand
that is missing?} missing. But then they used a multiplication table to deduce the
solution.

From here it will help if we have a times-table to reference: 
\begin{center}
\global\long\def\arraystretch{1.3}
 \setlength{\doublerulesep}{0pt} %
\begin{tabular}{r||2|2|2|2|2}
$\times$  & 0  & 1  & 2  & 3  & 4 \tabularnewline
\hline 
\hline 
0  & 0  & 0  & 0  & 0  & 0 \tabularnewline
\hline 
1  & 0  & 1  & 2  & 3  & 4 \tabularnewline
\hline 
2  & 0  & 2  & 4  & 1  & 3 \tabularnewline
\hline 
3  & 0  & 3  & 1  & 4  & 2 \tabularnewline
\hline 
4  & 0  & 4  & 3  & 2  & 1 \tabularnewline
%\hline 
\end{tabular}
\par\end{center}

What the Egyptians did was re-frame the division problem as multiplication,
then read off the answer. For example, $3\div 2=?$ re-framed as multiplication
becomes $3=?\cdot 2$. Now we look at the times table and see that 4
times 2 gives me 3, therefore $?=4$ and $3\div 2=4$. Remember division
is multiplication inverted. In other words, each division problem
is really just a multiplication problem in reverse. Next we will show
you a structure that does not have a multiplicative inverse.

\subsubsection*{4-hour clock arithmetic}

Consider a structure whose set of numbers consists of four elements
$\{0,1,2,3\}$, we will call this a 4-clock arithmetic. Similar to
the 5-clock structure it is possible to add these numbers, for example,
$1+1=2$, $1+2=3$, $0+1=1$ and likewise $1+3=0$. In this structure
we map the number 4 to 0, $5\rightarrow1,6\rightarrow2,\ldots$etc.
We can also consider multiplication take for example: 
\[
2+2+2+3=1
\]
Again we can utilize repeated addition to understand multiplication.
\begin{align*}
2+2+2+3 & =1\\
2\cdot3+3 & =1\\
(2+1)\cdot3 & =1\\
3\cdot3 & =1
\end{align*}
This system has a multiplicative identity. Does it have a multiplicative
inverse, can we divide? From here it will be helpful to reference
a times table in this 4-clock structure: 
\begin{center}
\global\long\def\arraystretch{1.3}
 \setlength{\doublerulesep}{0pt} %
\begin{tabular}{r||2|2|2|2}
$\times$  & 0  & 1  & 2  & 3 \tabularnewline
\hline 
\hline 
0  & 0  & 0  & 0  & 0 \tabularnewline
\hline 
1  & 0  & 1  & 2  & 3 \tabularnewline
\hline 
2  & 0  & 2  & 0  & 2 \tabularnewline
\hline 
3  & 0  & 3  & 2  & 1 \tabularnewline
%\hline 
\end{tabular}
\par\end{center}

Suppose we wanted to find $3\div2=?$. Previously we used the multiplicative
inverse to find the solution to a division problem, so we'll try that
same strategy again. we'll re-frame the problem as a multiplicative
one $3=?\cdot2$, now we want to multiply both sides by the multiplicative
inverse of 2, however we are not able to, because there is no number,
which we can multiply 2 by and get 1 in this 4-clock structure. There
does not exist a solution in the $4-$clock arithmetic to $3\div2$,
for more on this see the appendix on Euclidean division.

The reason that we get a field modulo $5$ but we don't get a field
modulo $4$ is that $4=2\cdot2$ whereas $5$ is a prime number, that
is, it is not the product of two integers smaller than itself. Thus
in division there always exists a unique remainder. The important
thing to take away from this is that clock or modulo arithmetic gives
us a number system in which we can add, subtract, multiply and divide
if and only if the modulus or number of hours on the clock is a prime
(also called irreducible) number.

\subsubsection*{Applying the idea of field extension to polynomials}

Suppose that we are in the real numbers with any polynomial of the
form 
\[
x^{3}+x^{2}+x+1
\]
and we want to factor this polynomial. It looks like $x=-1$ might
be a root. So we use long division to find 
\[
x^{3}+x^{2}+x+1=(x+1)(x^{2}+1)
\]
The polynomial $x^{2}+1$ cannot be factored in $\mathbb{R}[x]$,
just like the number $5$ cannot be factored in the number system
$\mathbb{Z}.$ Just as we found that setting the prime number $5$
equal to zero led to the fact that every other number in the system
has a multiplicative inverse and so the $5$-hour clock system is
a field, we can apply the same logic to polynomials in $\mathbb{R}\left[x\right]$,
i.e., we can set 
\[
x^{2}+1=0
\]
which then implies that we must set all polynomial multiples of $x^{2}+1$
equal to zero and see what happens. Well, if $g\left(x\right)\in\mathbb{R}\left[x\right]$
is any polynomial with real coefficients, then we can do long division
with remainder 
\[
g\left(x\right)\div\left(x^{2}+1\right)=?
\]
to get 
\[
g\left(x\right)=h\left(x\right)\cdot\left(x^{2}+1\right)+r\left(x\right)
\]
where $r\left(x\right)$ is a polynomial of degree less than two.
This equation says that in our system, $x^{2}+1=0$ so: 
\[
g\left(x\right)=r\left(x\right).
\]
Every element in this number system can be represented by a polynomial
of degree less than two, just like every integer can be represented
in $5$-clock arithmetic by either $0$, $1$, $2$, $3$, or $4$.

If $r\left(x\right)=0$ then $g\left(x\right)$ is a multiple of $x^{2}+1$
and so is also zero.  If $g\left(x\right)$ is not zero in this
system, either $r\left(x\right)$ is a polynomial of degree one or
a non-zero constant. If $r\left(x\right)$ is a non-zero constant
polynomial $a_{0}\in\mathbb{R}$, then it has a multiplicative inverse
in this system, namely $a_{0}^{-1}\in\mathbb{R}$. If $r\left(x\right)$
is a polynomial of degree one, then we can do long division with remainder
therefore there exists a $k(x)$ and a $b_{0}$ such that 
\[
\left(x^{2}+1\right)=k\left(x\right)\cdot r\left(x\right)+b_{0}
\]
where $b_{0}$ is a constant polynomial. Notice that $b_{0}\neq0$
since, if it were zero, $x^{2}+1$ would could factor in $\mathbb{R}\left[x\right]$.
So in this system 
\[
0=k\left(x\right)\cdot r\left(x\right)+b_{0}
\]
but we already know that $g\left(x\right)=r\left(x\right)$ so 
\[
0=k\left(x\right)\cdot g\left(x\right)+b_{0}.
\]
So dividing both sides by $-b_{0}$ and adding one we get 
%\[
%0=\left(-b_{0}^{-1}\cdot k\left(x\right)\right)\cdot g\left(x\right)-1,
%\]
\[1 = \underbrace{\left(-b_{0}^{-1}\cdot k\left(x\right)\right)}_{g(x)^{-1}} \cdot g\left(x\right) \]
that is $\left(-b_{0}^{-1}\cdot k\left(x\right)\right)$ is the multiplicative
inverse of $g\left(x\right)$. The notation for this system is 
\[
\mathbb{R}[x]/(x^{2}+1).
\]

In short we ''extend the field" by setting the un-factorable term to 0, from that it follows that
the polynomial 
\[
x\in\mathbb{F}=\mathbb{R}[x]/(x^{2}+1)
\]
is a root of the polynomial 
\[
y^{2}+1\in\mathbb{F}\left[y\right]
\]
since, substitution $x\in\mathbb{F}$ for $y$, we get $x^{2}+1\in\mathbb{F}$
and in $\mathbb{F}$, $x^{2}+1=0$. This would be a field extension
of $\mathbb{R}$, namely $\mathbb{F}$ is a field and $\mathbb{F}$
contains the field $\mathbb{R}$.\footnote{Effectively we are creating a new number, $x$, such that $x$ is the solution to $x^2+1=0$.}

For example multiplication in this field extension, $\mathbb{R}[x]/(x^{2}+1)$ is
\begin{align*}
(a+bx)(c+dx) & =(a+bx)c+(a+bx)dx\\
 & =ac+bcx+adx+bdx^{2}\\
 & =(ac-bd+bd\left(x^{2}+1\right))+(ad+bc)x\\
 & =(ac-bd)+(ad+bc)x
\end{align*}
Usually we replace the letter $x$ by the letter $i$ and call this
system the complex numbers. So we have obtained the complex numbers
as a field extension of the real number system. We are doing this
because it will turn out that we can split any polynomial up in this
way (i.e. find the splitting field).

\subsubsection*{Splitting field of $x^{3}-r_{1}x^{2}+r_{2}x-r_{3}$}

Suppose that $x^{3}-r_{1}x^{2}+r_{2}x-r_{3}$ is irreducible in some
system of polynomials $\mathbb{F}[x]$ with coefficients in some field
$\mathbb{F}$ about which we know nothing. ($\mathbb{F}$ can't be
the field of real numbers because we have seen above that any polynomial
of degree three in $\mathbb{R}[x]$ has at least one real root and
so can be factored in $\mathbb{R}[x]$.)

What we want to do is construct a clock arithmetic where 
\[
x^{3}-r_{1}x^{2}+r_{2}x-r_{3}\equiv0,
\]
We will call this new field $\mathbb{G}$ and write $\mathbb{G}[y]$
for the system of polynomials with coefficients in $\mathbb{G}$ .
As we saw above, the polynomial $f(y)=y^{3}-r_{1}y^{2}+r_{2}y-r_{3}$
has the root $x\in\mathbb{G}$ since $f(x)=x^{3}-r_{1}x^{2}+r_{2}x-r_{3}= 0$
therefore $x$ is root of $y^{3}-r_{1}y^{2}+r_{2}y-r_{3}$.\footnote{Since $x$ is not divisible by $y^{3}-r_{1}y^{2}+r_{2}y-r_{3}$, it
cannot be a multiple of it.} For the sake of convenience with notation, we set $x=s_{1}$. We can then
reduce the degree of the initial polynomial by long division\footnote{Long division of polynomials is possible provided the coefficients are in a field.  See the appendix on Euclidean division.} 
so that $y^{3}-r_{1}y^{2}+r_{2}y-r_{3}$ becomes 
\[
(y-s_{1})(y^{2}+by+c)
\]
where $s_{1}$, $b$ and $c$ are some values in $\mathbb{G}[x]$.
We complete the square on the quadratic term to find the other remaining
two roots. 
\begin{align*}
y^{2}+by+c & =0\\
\left(y+\frac{b}{2}\right)^{2}-\frac{b^{2}}{4}+c & =0\\
\left(y+\frac{b}{2}\right)^{2} & =\frac{b^{2}}{4}-c\\
\left(y+\frac{b}{2}\right)^{2} & =\frac{b^{2}-4c}{4}\\
\abs{y+\frac{b}{2}} & =\sqrt{\frac{b^{2}-4c}{4}}\\
y+\frac{b}{2} & =\pm\sqrt{\frac{b^{2}-4c}{4}}\\
y+\frac{b}{2} & =\pm\frac{\sqrt{b^{2}-4c}}{2}\\
y & =\frac{-b}{2}\pm\frac{\sqrt{b^{2}-4c}}{2}\\
y & =\frac{-b\pm\sqrt{b^{2}-4c}}{2}
\end{align*}
If we can solve the equation 
\[
y^{2}-\left(b^{2}-4c\right)=0
\]
for some $y=s_{0}\in\mathbb{G}$ then for simplification we write
\[
s_{0}=\sqrt{b^{2}-4c}.
\]
Then $y^{2}+by+c$ can be factored in $\mathbb{G}\left[y\right]$
and we are done. Just let $s_{2}=\frac{-b+s_{0}}{2}$ and $s_{3}=\frac{-b-s_{0}}{2}$,
therefore the splitting field in $\mathbb{G}[y]$ is
\[
y^{3}-r_{1}y^{2}+r_{2}y-r_{3}=(y-s_{1})(y-s_{2})(y-s_{3}).
\]


If $y^{2}+by+c$ is not reducible, i.e., if we cannot solve $y^{2}-(b^{2}-4c)$
in $\mathbb{G}[y]$, then we need to expand the field again. This time
we set 
\[
y^{2}+by+c\equiv0
\]
and call this new field 
\[
\mathbb{J}=\mathbb{G}\left[y\right]/\left(y^{2}+by+c\right).
\]
Then just like before $y\in\mathbb{G}$ is a root of the polynomial
\[
z^{2}+bz+c\in\mathbb{J}\left[z\right].
\]
Again for convenience set $y=s_{2}$. Thus after long division of
$(z^{2}+bz+c)\div(z-s_{2})$, we will have a first degree polynomial,
so we will not need to repeat the procedure again.

To summarize: We split 
\[
z^{3}-r_{1}z^{2}+r_{2}z-r_{3}
\]
into 
\[
(z-s_{1})(z-s_{2})(z-s_{3})
\]
and $s_{1},s_{2},s_{3}$ are elements of this field $\mathbb{J}$.
This new field $\mathbb{J}$ may look very different than my original
field, $\mathbb{F}$. We may have had to make two field extensions
\[
\mathbb{F}\subseteq\mathbb{G}\subseteq\mathbb{J}
\]
in order to be able to factor $x^{3}-r_{1}x^{2}+r_{2}x-r_{3}$ completely.



\subsection*{Symmetric Polynomials}

A symmetric polynomial of several variables is a polynomial whose variables can be interchanged in any way without affecting the polynomial. For example, 
\[
x+y
\]
is symmetric, if we switch $x$ with $y$ the expression
$y+x$ is equivalent. Likewise 
\[
x+y+z
\]
is also symmetric, we may replace $x$ with $z$ and $z$ with $x$
(or any other combination of variables) and maintain equivalency.
However 
\[
x^{2}+y
\]
is not symmetric alternating the variables results in 
\[
y^{2}+x\not\equiv x^{2}+y
\]
The left hand side is not equivalent to the right hand side.

The polynomial $(z-s_{1})(z-s_{2})(z-s_{3})$ is symmetric in the three variables $s_1, s_2, s_3$. E.g.,

We know that in $\mathbb{J}[z]$ all the usual properties of addition
and multiplication hold (except multiplicative inverse). Therefore
there is nothing holding me back from distributing and multiplying:
\[
(z-s_{1})(z-s_{2})(z-s_{3})=z^{3}-(s_{1}+s_{2}+s_{3})z^{2}+(s_{1}s_{2}+s_{1}s_{3}+s_{2}s_{3})z-s_{1}s_{2}s_{3}
\]
Inspecting the right hand side of the previous equation notice how $s_{1}$ and $s_{2}$
could interchange positions and maintain an equivalent expression,
likewise with $s_{3}$. So each of the coefficients on the right-hand
side 
\begin{align*}
e_{1}\left(s_{1},s_{2},s_{3}\right) & =s_{1}+s_{2}+s_{3}\\
e_{2}\left(s_{1},s_{2},s_{3}\right) & =s_{1}s_{2}+s_{1}s_{3}+s_{2}s_{3}\\
e_{3}\left(s_{1},s_{2},s_{3}\right) & =s_{1}s_{2}s_{3}
\end{align*}
is a polynomial whose value is unchanged if the three variables $s_{1},s_{2},s_{3}$
are permuted in any way. These polynomials 
\[
p\left(s_{1},s_{2},s_{3}\right)
\]
are symmetric.

The three polynomials 
\begin{align*}
e_{1} & =s_{1}+s_{2}+s_{3}\\
e_{2} & =s_{1}s_{2}+s_{1}s_{3}+s_{2}s_{3}\\
e_{3} & =s_{1}s_{2}s_{3}
\end{align*}
are called elementary symmetric polynomials in the three variables
$s_{1},s_{2},s_{3}$ . In the same way, the $n$ coefficients of the
polynomial 
\[
(x-s_{1})(x-s_{2})\cdots(x-s_{n})=x^{n}-e_{1}(s_{1},\ldots,s_{n})x^{2}+ \cdots\pm e_{n-1}(s_{1},\ldots,s_{n})x\mp e_{n}(s_{1},\ldots,s_{n})
\]
are called the elementary symmetric polynomials in the $n$ variables
$s_{1},\ldots,s_{n}$.

Vi�te proved that every symmetric polynomial 
\[
p\left(s_{1},\ldots,s_{n}\right)
\]
can be written as a polynomial 
\[
q\left(e_{1},\ldots,e_{n}\right)
\]
whose ``variables' are elementary symmetric polynomials in the variables
$s_{1},\ldots,s_{n}$. Vi�te discovered that the necessary formulas
to prove this only utilize the basic properties of addition, subtraction,
multiplication, and division, so that they apply for coefficients
in any of the fields we might be interested in. Said otherwise, if
the coefficients of $p\left(s_{1},\ldots,s_{n}\right)$ lie in some
number system $\mathbb{S}$, then the coefficients in the polynomial
$q\left(e_{1},\ldots,e_{n}\right)$ lie in the same number system
$\mathbb{S}$.

\paragraph*{Example:}

Using Vi�te's formulas to rewrite 
\[
s_{1}^{2}+s_{2}^{2}+s_{3}^{2}
\]
as the previously defined elementary symmetric polynomials.  First notice how $s_{1}^{2}+s_{2}^{2}+s_{3}^{2}$ is symmetric in
$s_{1},s_{2},s_{3}$. We can interchange any of the variables and we will
have an equivalent expression.  Next we will need the sum of squares: 
\[
e_{1}^{2}=s_{1}^{2}+s_{2}^{2}+s_{3}^{2}+2s_{1}s_{2}+2s_{1}s_{3}+2s_{2}s_{3}
\]
Now we need to remove the cross terms, $e_{2}$ should work nicely
for that 
\[
e_{1}^{2}-2e_{2}=s_{1}^{2}+s_{2}^{2}+s_{3}^{2}+2s_{1}s_{2}+2s_{1}s_{3}+2s_{2}s_{3}-2(s_{1}s_{2}+s_{1}s_{3}+s_{2}s_{3})
\]
after zeroing out terms we arrive at 
\[
e_{1}^{2}-2e_{2}=s_{1}^{2}+s_{2}^{2}+s_{3}^{2}.
\]
\mbox{%
%
}\\

Returning to the polynomial in question, $z^{3}-r_{1}z^{2}+r_{2}z-r_{3}$,
the elementary symmetric polynomials $e_{1},e_{2},e_{3}$ are the
same as the coefficients, $r_{1},r_{2},r_{3}$ respectively. For the sake of argument let us
suppose the coefficients $r_{i}$ are real.\footnote{As shown this would then imply $\mathbb{J}=\mathbb{G}=\mathbb{C}$} This means we can relate
the roots in our strange field $\mathbb{J}$ with the coefficients
of the original polynomial that lie in $\mathbb{R}$. Since the $r_{1},r_{2},r_{3}$
lie in $\mathbb{R}$, i.e. the $e_{1},e_{2},e_{3}$ lie in $\mathbb{R}$.
But we still can't make the leap to $s_{1},s_{2},s_{3}$ lying in $\mathbb{R}$.
For example suppose 
\begin{align*}
s_{1}=1-i\\
s_{2}=1+i\\
s_{3}=1
\end{align*}
Then the elementary symmetric polynomials would become 
\begin{align*}
e_{1}=3\\
e_{2}=4\\
e_{3}=2
\end{align*}

The elementary symmetric polynomials are real, however only one of
the three roots is real valued. \\

In our proof of the Fundamental Theorem of Algebra, we will need to
use induction on polynomial equations whose coefficients are symmetric
polynomials. For example, if we start with 
\[
z^{3}-r_{1}z^{2}+r_{2}z-r_{3}=(z-s_{1})(z-s_{2})(z-s_{3})
\]
with the left-hand side in $\mathbb{R}\left[z\right]$ and the right-hand
side in $\mathcal{\mathbb{J}}\left[z\right]$, we will want to form
the polynomial 
\[
G_{t}(z)=(z-s_{1}-s_{2}-ts_{1}s_{2})(z-s_{1}-s_{3}-ts_{1}s_{3})(z-s_{2}-s_{3}-ts_{2}s_{3}).
\]
Where $t$ is an external variable that only takes real values.  When we expand $G_{t}(z)$ we will get a polynomial in the variable
$z$ whose coefficients are polynomials in $s_{1},s_{2},s_{3}$ and
those coefficient polynomials in the $s_{1},s_{2},s_{3}$ are symmetric
polynomials in the $s_{1},s_{2},s_{3}$ and their coefficients are
in the number system 
\[
\mathbb{S}=\mathbb{Z}\left[s\right].
\]
Let's check. Here is what we get:


\begin{flalign*}
G_t(z) &= z^3 \\
&\hphantom{{}=x} {\color{green} - z^2 ( 2 s_1 + 2 s_2 + 2 s_3 + s_1 s_2 t + s_1 s_3 t + s_2 s_3 t )} &\\
&\hphantom{{}=x+x} {\color{blue} + z( s_1 s_2 s_3^2 t^2 + s_1 s_2^2 s_3 t^2 + s_1^2 s_2 s_3 t^2 + s_1 s_2^2 t  + s_1 s_3^2 t  + s_2 s_3^2 t  + s_1^2 s_2 t  }  \\
&\hphantom{{}=x+x+x+x} {\color{blue} + s_1^2 s_3 t  + s_2^2 s_3 t + 6 s_1 s_2 s_3 t  + s_1^2  + s_2^2  + s_3^2  + 3 s_1 s_2  + 3 s_1 s_3  + 3 s_2 s_3 )} &\\
&\hphantom{{}=x+x+x} {\color{red} -s_1^2 s_2^2 s_3^2 t^3   - 2 s_1 s_2^2 s_3^2 t^2 - 2 s_1^2 s_2 s_3^2 t^2 - 2 s_1^2 s_2^2 s_3 t^2  - s_1^2 s_2^2 t - s_1^2 s_3^2 t - s_2^2 s_3^2 t} \\
&\hphantom{{}=x+x+x+x} {\color{red} - 3 s_1 s_2 s_3^2 t - 3 s_1 s_2^2 s_3 t - 3 s_1^2 s_2 s_3 t  - s_1 s_2^2 - s_1 s_3^2 - s_2 s_3^2 - s_1^2 s_2 - s_1^2 s_3}\\
&\hphantom{{}=x+x+x+x} {\color{red} - s_2^2 s_3 - 2 s_1 s_2 s_3 }
\end{flalign*}

Notice how we could switch all the $s_{1}$ with $s_{2}$ (likewise
with $s_{1}$ and $s_{3}$ or $s_{2}$ and $s_{3}$) and we would end
up with an equivalent equation. Thus the polynomial viewed as a function
of $s_{1},s_{2},s_{3}$ is symmetric and $G_{t}(z)$ is symmetric
in the coefficients $s_{1},s_{2},s_{3}$. Since $G_{t}(z)$ is symmetric
then we can use Vi�te's formulas and rewrite $G_{t}(z)$ in terms of
its elementary symmetric polynomials.\footnote{This is known as the Fundamental Theorem of Symmetric Polynomials,
the proof of which is a bit beyond this paper.  Please see the excellent paper by Ben Blum-Smith and Samuel Coskey, ''The Fundamental Theorem on Symmetric Polynomials: History's First Whiff of Galois Theory'': \url{https://arxiv.org/abs/1301.7116} for more exposition.} Recall that the first three elementary symmetric polynomials are 
\begin{align*}
e_{1} & =s_{1}+s_{2}+s_{3}\\
e_{2} & =s_{1}s_{2}+s_{1}s_{3}+s_{2}s_{3}\\
e_{3} & =s_{1}s_{2}s_{3}
\end{align*}

Using the elementary symmetric polynomials we can rewrite


\begin{flalign*}
G_t(z) &= z^3 \\
&\hphantom{{}=x} {\color{green}- z^2 ( 2 e_1 + e_2 t )} &\\
&\hphantom{{}=x+x} {\color{blue}+ z( e_1 e_3 t^2 + e_1 e_2 t  + e_1^2 - 2e_2  + 3 e_2 ) }&\\
&\hphantom{{}=x+x+x} {\color{red}-e_3^2 t^3   - 2 e_2 e_3 t^2  - (e_2^2 - 2e_1e_3) t  - 3 e_1 e_3 t  - e_1 e_2 + e_3}
\end{flalign*}

The coefficients of the elementary symmetric polynomials are real
valued. Furthermore the elementary symmetric polynomials are real
valued. Since $G_t(z)$ is a polynomial in the variables $z$ and $t$ with coefficients that are symmetric in the variables $s_1, s_2, s_3$, Vi{\'e}te's theorem tells us that $G_t(z)$ may be rewritten as a polynomial in the variables $z$ and $t$ with coefficients in the number system $S=\mathbb{Z}[e_1,e_2,e_3]$.  Now recall that the $e_1, e_2, e_3$ are just coefficients of the original polynomial,

\[
x^{3}-r_{1}x^{2}+r_{2}x-r_{3}
\]

i.e., $e_1=r_1, e_2=r_2, e_3=r_3$.  Since $e_1, e_2, e_3$ are real numbers, that means that $S=\mathbb{Z}[e_1, e_2, e_3]$ is a subset of $\mathbb{R}$, i.e., $G_t(z)$ is a polynomial in the variables $z$ and $t$ with real valued coefficients!



%Recall that the $e_{1},e_{2},e_{3}$ relate to the original
%polynomial, 
%\[
%x^{3}-r_{1}x^{2}+r_{2}x-r_{3}
%\]
%Not only did $e_{1},e_{2},e_{3}$ relate to the roots, but they related
%to the coefficients, i.e., 
%\begin{align*}
%e_{1}=r_{1}\\
%e_{2}=r_{2}\\
%e_{3}=r_{3}
%\end{align*}
%
%Remember that $r_{1},r_{2},r_{3}$ are real valued, $\mathbb{R}$.
%Now $G_{t}(z)$ may be rewritten as a polynomial in the elementary
%symmetric polynomials with coefficients in $\mathbb{S}=\mathbb{Z}\left[s\right]$,
%which are equivalent to the coefficients in the original polynomial.
%Thus the coefficients of $G_{t}(z)$ must be real valued, that is,
%must lie in $\mathbb{R}\left[s\right].$

\subsubsection*{Complex number review}

Recall that to take the conjugate of a complex number you just switch the sign of the imaginary part  (e.g. if $z=1+2i$
then the conjugate of $z$ is $\bar{z}=1-2i$). So a complex number
$a+bi$ is in fact a real number if and only if $b=0$, that is, if
and only if 
\[
a+bi=\overline{a+bi}.
\]
Also, in the complex numbers, the sum of conjugates equals the conjugate
of the sum and the product of conjugates equals the conjugate of the
product.

Something neat also happens when you multiply a complex number with
its conjugate. Let's let $z=a+bi$ be any complex number, then the
conjugate is $\bar{z}=a-bi$ and if we multiply them together we get
\[
z\bar{z}=a^{2}+abi-abi-(bi)^{2}
\]
which becomes 
\[
z\bar{z}=a^{2}+b^{2}
\]
which is a real number, i.e., the imaginary part is 0 (or there is
no imaginary part). Furthermore it is a positive real number.

\section*{The Fundamental Theorem of Algebra}

So now we are ready to begin our proof of the Fundamental Theorem
of Algebra, namely the theorem that says that any polynomial of degree
$d>0$ with complex coefficients has at least one complex root.  This proof originally appeared in \citeA{samuel2013algebraic}, pg. 45.

\subsection*{Reduction to polynomials with real coefficients}

We start with any polynomial of degree $d$ with coefficients which
are complex numbers. We can always divide through by the coefficient
of $x^{d}$ to reduce our polynomial to one of the form 
\[
p\left(x\right)=x^{d}+a_{d-1}x^{d-1}+\ldots+a_{1}x+a_{0}\in\mathbb{C}\left[x\right].
\]
We need to show that there is a complex number $z_{1}$ that is a
root of this polynomial. We start by forming the polynomial 
\[
P\left(x\right)=\left(x^{d}+a_{d-1}x^{d-1}+\ldots+a_{1}x+a_{0}\right)\left(x^{d}+\overline{a_{d-1}}x^{d-1}+\ldots+\overline{a_{1}}x+\overline{a_{0}}\right).
\]
Multiplying out the right-hand side we get 
\[
\begin{array}{c}
P\left(x\right)=x^{2d}+\left(a_{d-1}+\overline{a_{d-1}}\right)x^{2d-1}+\\
\ldots\\
+\left(a_{1}\overline{a_{0}}+a_{0}\overline{a_{1}}\right)+a_{0}\overline{a_{0}}.
\end{array}
\]
Now each coefficient in $P\left(x\right)$ has the property that its
conjugate is itself\textendash that just follows from the fact that
the conjugate of a sum is the sum of the conjugates and the conjugate
of a product is the product of the conjugates. So all the coefficients
in the polynomial $P\left(x\right)$ are real! Also, it suffices to
find a complex root $z_{1}$ of $P\left(x\right)$ since such a $z_{1}$
is either a root of 
\[
x^{d}+a_{d-1}x^{d-1}+\ldots+a_{1}x+a_{0}
\]
or it is a root of 
\[
x^{d}+\overline{a_{d-1}}x^{d-1}+\ldots+\overline{a_{1}}x+\overline{a_{0}}.
\]
But in this second case 
\[
\begin{array}{c}
z_{1}^{d}+\overline{a_{d-1}}z_{1}^{d-1}+\ldots+\overline{a_{1}}z_{1}+\overline{a_{0}}=0\\
\overline{z_{1}^{d}+\overline{a_{d-1}}z_{1}^{d-1}+\ldots+\overline{a_{1}}z_{1}+\overline{a_{0}}}=\overline{0}=0\\
\overline{z_{1}}^{d}+a_{d-1}\overline{z_{1}}^{d-1}+\ldots+a_{1}\overline{z_{1}}+a_{0}=0
\end{array}
\]
so we can put $z_{0}=\overline{z_{1}}$ . So it suffices to show that
every polynomial with real coefficients has a complex root.

\subsection*{Roots of polynomials with real coefficients}

Now suppose that we have any polynomial 
\[
x^{d}+r_{d-1}x^{d-1}+\ldots+r_{1}x+r_{0}
\]
with real coefficients. We can always factor 
\[
d=2^{k}\cdot m
\]
with $m$ odd. We will show that for any value $k$ there is at least
one complex root of this polynomial. We will do this by inducting
on $k$.

\subsubsection*{Base Case}

For the base case $k=0$, the polynomial has odd degree, and, as we
have explored above, the end behavior of odd polynomials with positive
leading coefficient is $x\rightarrow-\infty$ the value of the polynomial
goes to negative infinity and as $x\rightarrow\infty$ the value of the
polynomial goes to positive infinity. Recall that a polynomial is continuous,
there are no jumps, therefore the polynomial must cross the $x-$axis
at some place in between $(-\infty,\infty)$ therefore there is at
least one real root. (We give a more rigorous proof in the Appendix.)
This real root of course counts as our complex root since every real
number is a complex number. This argument takes care of any polynomials
with odd degree.

\subsubsection*{Induction Step}

Now for the induction step, where we will use the induction hypothesis
that every polynomial with real coefficients and degree $d=2^{k-1}m'$
(where $m'$ is odd) has at least one complex root.

As we have seen above, there is $some$ field $\mathbb{F}\supseteq\mathbb{C}\supseteq\mathbb{R}$
so that we can factor 
\[
x^{d}+r_{d-1}x^{d-1}+\ldots+r_{1}x+r_{0}=\left(x-x_{1}\right)\cdot\ldots\cdot\left(x-x_{d}\right)
\]
with all the $x_{i}\in\mathbb{F}$.

Here comes the ingenious step! Let $s$ be an arbitrary real number
and let $y_{s,i,j}=x_{i}+x_{j}+sx_{i}x_{j}$, where $1\leq i<j\leq d$.
Now define: 
\[
G_{s}(x)=(x-y_{s,1,2})(x-y_{s,1,3})\cdots(x-y_{s,1,d})(x-y_{s,2,3})\cdots(x-y_{s,2,d})\cdots(x-y_{s,d-1,d})
\]
This may be succinctly written as: 
\[
G_{s}(x)=\Pi_{1\leq i<j\leq d}(x-y_{s,i,j})
\]

$G_s(x)$ is a polynomial in the variables $x$ and $s$ whose coefficients are in the number system $\mathbb{Z}[x_1, \ldots, x_d]$.  But each of those coefficients is a symmetric polynomial in $x_1, \ldots, x_d$ since it doesn't change under any permutation of the $x_i$.  So, as in the previous example, by Vi{\'e}te's theorem each coefficient in $G_s(x)$ is a polynomial in the variables $x$ and $s$ with coefficients in the elementary symmetric functions $e_1, \ldots, e_d$ of $x_1, \ldots, x_d$.  Again as in the previous example, $e_1, \ldots, e_d$ are real numbers!  So $G_s(x)$ is a polynomial in the variables $x$ and $s$ with coefficients that are real numbers!





%The coefficients of $x$ in $G_{s}(x)$ are polynomials in $x_{1},\ldots,x_{d}$
%whose coefficients are in the number system $\mathbb{Z}\left[x_i\right]$.
%But each of those coefficients is a symmetric polynomial in $x_{1},\ldots,x_{d}$
%since it doesn't change under any permutation of the $x_{i}$ . So
%by Vi�te's theorem, each coefficient of $x$ in $G_{s}(x)$ is a polynomial
%in the elementary symmetric functions in $x_{1},\ldots,x_{d}$ with
%coefficients in $\mathbb{Z}\left[x_i\right]$. But those elementary
%elementary symmetric functions in $x_{1},\ldots,x_{d}$ are just $r_{d-1},\ldots,r_{0}$!
%And $r_{d-1},\ldots,r_{0}$ are all real numbers! So each coefficient
%of $x$ in $G_{s}(x)$ lies in the real number system, $\mathbb{R}$.
%
%But what is the degree of $G_{s}(x)$? That is, how many terms there
%are in $G_{s}(x)$? 



But what is the degree of $G_s(x)$ in the variable $x$?  We get exactly one term for each way of choosing
two distinct numbers out of the set $\left\{ 1,\ldots,d\right\} $.
You may recognize this as the 'choose number' $\binom{d}{2}$. 

Recall that $\binom{d}{2} = \frac{d(d-1)}{2}$ (see appendix for derivation), then by substitution:
\[
\frac{d\cdot\left(d-1\right)}{2}=\frac{2^{k}\cdot m\cdot\left(2^{k}\cdot m-1\right)}{2}=2^{k-1}\cdot m\cdot\left(2^{k}\cdot m-1\right).
\]
But $m$ is odd and, since $k>0$, $\left(2^{k}\cdot m-1\right)$
is also odd and so $m'=m\cdot\left(2^{k}\cdot m-1\right)$ is also
odd. So, by the induction hypothesis, for any fixed real number $s$,
the polynomial $G_{s}(x)\in\mathbb{R}\left[x\right]$ has a complex
root! So, for each real number $s$, at least one of the $y_{s,i,j}$
must be a complex number!

We do not know which 
\[
x_{i}+x_{j}+sx_{i}x_{j}
\]
is complex for a given $s$ but there are infinitely many real numbers
$s$ and only finitely many pairs $ij$ so there must be some $ij$
such that 
\[
x_{i}+x_{j}+sx_{i}x_{j}
\]
is a complex number, $z_s$, for an infinite number of real numbers $s$. Pick two of those, say
$s'$ and $s''$ . So we have a system of two linear equations 
\[
\begin{array}{c}
\left(x_{i}+x_{j}\right)+s'x_{i}x_{j}=z'\in\mathbb{C}\\
\left(x_{i}+x_{j}\right)+s''x_{i}x_{j}=z''\in\mathbb{C}
\end{array}
\]
in two unknowns $b=\left(x_{i}+x_{j}\right)$ and $c=x_{i}x_{j}$.
Solving the system of two linear equations in two unknowns, we conclude
that since $s'$, $s''$, $z'$, and $z''$ all lie in the complex
number system, so do the unknowns $b=\left(x_{i}+x_{j}\right)$ and
$c=x_{i}x_{j}$. Finally consider the quadratic equation 
\[
x^{2}-(x_{i}+x_{j})x+x_{i}x_{j}=x^{2}-b\cdot x+c=0
\]
with complex coefficients. Applying the quadratic formula, the solutions
are 
\[
x=\frac{b\text{\textpm}\sqrt{b^{2}-4c}}{2}.
\]
On the other hand 
\begin{align*}
x^{2}-(x_{i}+x_{j})x+x_{i}x_{j} & =\left(x-x_{i}\right)\left(x-x_{j}\right)\\
 & =\left(x-\frac{b\text{\textpm}\sqrt{b^{2}-4c}}{2}\right)\left(x-\frac{b\mp\sqrt{b^{2}-4c}}{2}\right)
\end{align*}
So 
\[
\begin{array}{c}
x_{i}=\frac{b\pm\sqrt{b^{2}-4c}}{2}\\
x_{j}=\frac{b\mp\sqrt{b^{2}-4c}}{2}.
\end{array}
\]
We know $b$ and $c$ are complex so to show that $x_j$ and/or $x_i$ is complex we only need to show that the square root of a complex
number $z=b^{2}-4c$ is again a complex number.

% So to show that $x^{d}+r_{d-1}x^{d-1}+\ldots+r_{1}x+r_{0}$ has a complex root,

\subsubsection*{Proof the square root of a complex number is complex}

Write $z$ in polar coordinates as $p=r\cdot(\cos(\theta)+i\sin(\theta))$.
We seek a complex number $w$ such that $w^{2}=p$. Using the sum
of angles formula from trigonometry, we compute 
\[
\begin{array}{c}
\left(r^{1/2}\cdot(\cos(\theta/2)+i\sin(\theta/2))\right)^{2}=\\
\left(r^{1/2}\right)^{2}\cdot((\cos(\theta/2)+i\sin(\theta/2)))^{2}=\\
r\cdot\left(\left(\cos^{2}\left(\theta/2\right)-\sin^{2}\left(\theta/2\right)\right)+2i\sin(\theta/2)\cdot\cos(\theta/2)\right)\\
r\cdot(\cos(\theta)+i\sin(\theta))=p.
\end{array}
\]
So the complex number 
\[
w=r^{1/2}\cdot(\cos(\theta/2)+i\sin(\theta/2))
\]
is the square root of the complex number $p$.  Likewise both $x_i$ and $x_j$ are complex.

So we have finished the proof of the Fundamental Theorem of Algebra.
Namely we have shown that every polynomial with coefficients which
are complex numbers 
\[
p\left(x\right)=x^{d}+a_{d-1}x^{d-1}+\ldots+a_{1}x+a_{0}\in\mathbb{C}\left[x\right]
\]
has at least one root that is a complex number. So we are almost done
with the story. But we can say a bit more.

\subparagraph{Every single variable polynomial of degree $d$ in $\mathbb{C}\left[x\right]$
has exactly $d$ roots}

Let's prove this as a corollary of the Fundamental Theorem of Algebra.
Set: 
\[
f_{0}(x)=a_{d}x^{d}+a_{d-1}x^{d-1}+\cdots+a_{1}x+a_{0}
\]
where the $a_{i}$ are complex. We know by the Fundamental Theorem
of Algebra that $f_{0}(x)$ has at least one complex root, let's call
it $r_{1}$. But since $\mathbb{C}$ is a field, we can divide polynomials
in $\mathbb{C}\left[x\right]$ by the linear polynomial $x-r_{1}$
so that we can write 
\[
f_{0}(x)=(x-r_{1})\cdot\underbrace{f_{1}(x)}_{\textrm{\text{quotient}}}+\underbrace{b_{0}}_{\textrm{remainder}}.
\]
Substituting we have 
\[
0=f_{0}\left(r_{1}\right)=\left(r_{1}-r_{1}\right)f_{1}\left(r_{1}\right)+b_{0}=b_{0}.
\]
So in fact 
\[
f_{0}(x)=(x-r_{1})\cdot f_{1}\left(x\right)
\]
with $f_{1}\left(x\right)\in\mathbb{C}\left[x\right].$\\

The degree of $f_{0}(x)$ was $d$, therefore the degree of $f_{1}(x)$
must be $d-1$, since 
\[
\underbrace{f_{0}(x)}_{\text{degree }d}=\underbrace{(x-r_{1})}_{\text{degree }1}f_{1}(x)
\]
and exponents add.\\

Now we apply the Fundamental Theorem of Algebra to $f_{1}(x)$ this
gives us a root we will call $r_{2}$. We divide $f_{1}(x)$ by $(x-r_{2})$
to get $f_{2}(x)$. Again the remainder must be zero and the degree
of $f_{2}(x)$ will be $d-2$. Now we have found two roots and we've
reduced the polynomial by two degrees. Therefore if we repeat this
process $d$ times we will have exactly $d$ roots and we will have
reduced the polynomial to degree $d-d=0$, for which no roots will
exist. Thus we have found exactly $d$ roots for a polynomial of degree
$d$. In fact in $\mathbb{C}\left[x\right]$ we have the complete
factorization 
\[
p\left(x\right)=x^{d}+a_{d-1}x^{d-1}+\ldots+a_{1}x+a_{0}=\left(x-r_{1}\right)\cdot\cdots\cdot\left(x-r_{d}\right).
\]
Another way to say this is that the only irreducible polynomials in
$\mathbb{C}\left[x\right]$ are linear polynomials $ax+b$.\pagebreak{}

\section*{Conclusion}

Analogies often play critical roles in mathematics. This paper explored the analogies between polynomials and integers. We reviewed
prime factorization and the fundamental theorem of arithmetic. This
led us to find an analogue of the fundamental theorem of arithmetic
for polynomials, which turned out to be the key to to establish the
fundamental theorem of algebra!

%We often focus on how things are different, but sometimes it is worthwhile to analyze how things are the same.  By looking at how systems are similar we can analyze the new system by simply adapting the previous tools.  In this case, we adapted modular arithmetic to find a splitting field.  In the process of adapting tools we generalize their function, thus making the tool more useful.  

%The original polynomial $C(x)$ had complex coefficients, which when
%multiplied by its conjugate $\bar{C(x)}$ produced a polynomial with
%real coefficients, we called this polynomial $R(x)$. The goal was
%to show that this polynomial $R(x)$ had at least one complex root.
%We found that complex root by inducting on the highest power of 2 that divides the degree
%of the polynomial $R(x)$. Induction enabled us to assume that polynomials
%of \char`\"{}lesser'' degree do have at least one complex root. We
%were able to construct a polynomial, $G_{t}(z)$ that not only satisfied
%the induction hypothesis, but whose roots were the roots of $R(x)$.
%So when we found $G_{t}(z)$'s complex root, we also found $R(x)$'s
%complex root. We then used this result to reduce the degree of the
%initial polynomial, and find the complex roots. Using this result
%we were able to find $n$ roots for a $n^{\text{th}}$ degree polynomial.
%In other words, we were able to decompose any polynomial into its
%constituent parts. Just as any integer may be decomposed into its
%prime factors, polynomials may be completely decomposed into linear
%factors in the complex plane. So the Fundamental Theorem of Arithmetic
%for the Integers has a complete analogue in the factoring of polynomials into irreducible polynomials, leading directly to the Fundamental Theorem of Algebra.

\begin{appendices}

\section{Division}

Division in the language of abstract algebra is often defined as the
solution, $s$ to the equation $a\cdot s=t$.\footnote{Technically this would be the definition for left division, right
division could be defined as the solution, $k$ to the equation $k\cdot a=t$.} This is similar to how the first peoples framed division. When we
use the word division we often think of \char`\"{}chunking'' values
or items, yet considering division in this way is only true for integers.
When we compute division we are calculating the quotient and the remainder.
The reason we know we are able to do this for any integer is attributable
to Euclidean division.

\subsection{Euclidean Division}
\subsubsection{Euclidean division for integers}
Euclidean division is a theorem that states: %\cite{burton2006elementary}

Given two positive integers $a \geq b$, there is a unique positive integer $q$ and non-negative integer $r$ such that $r < b$ and
$a = q\cdot b + r$

\subsubsection{Proof of Euclidean division for integers}
We will induct on $n$, such that $a \leq n$

\paragraph{Base case $n=1$:}
$1 = 1 \cdot 1 + 0$

\paragraph{Induction step:}  Assume there is a unique positive integer $q$ and non-negative integer $r$ such that $r < b$ and $a = q\cdot b + r$ for $a=n-1$.  This implies $n-1 = q \cdot b + r$, which may be rewritten as $n = q \cdot b + (r+1)$.

The Euclidean division theorem tells us that for any integer division problem, not only
does a quotient and remainder exist but there is a unique quotient
and remainder. This is a very important fact, one that is not true
for the $4-$clock number system where we saw a specific instance
where a solution did not exist. This theorem is often taken for granted
because we have had so much experience with division of integers,
but the contrast with $4-$clock arithmetic highlights its importance.

\subsubsection{Euclidean division of polynomials}
Euclidean division has also been generalized for division of polynomials.  It is typically stated as:

Given two polynomials $a(x)$ and $b(x)$ such that $deg( a(x)) \geq deg( b(x))$, there are unique polynomials $q(x)$ and $r(x)$
such that $deg( r(x)) < deg( b(x))$ and
$a(x) = q(x)\cdot b(x) + r(x)$ 

\subsubsection{Proof of Euclidean division of polynomials}
We will induct on $n$ such that $deg(a(x)) \leq n$.

\paragraph{Base case $n=1$:}
Then $deg(a(x))=1$, this implies $b(x)= b_0 =$ constant.  Further, $a_1 x + a_0 = b_0 q(x) + r(x)$ implies $q(x) = a_1 b_0^{-1} x + a_0 b_0^{-1}$ and $r(x)=0$.  Then $deg(r(x))=0<deg(b(x))=1$.

\paragraph{Induction step:}
For $a'(x)$ with degree $n-1$ and $b(x)$ with $deg(b(x))\leq n-1$, assume there are unique polynomials $q'(x)$ and $r'(x)$
such that  $deg( r'(x)) < deg( b(x))$ and
$a'(x) = q'(x)\cdot b(x) + r'(x)$.

We start by subtracting $\frac{a_n}{b_m} \cdot x^{n-m} \cdot b(x)$ from $a(x)$, where $deg(a(x))=n$ and $deg(b(x))=m \leq n-1$.
\begin{flalign*}
a(x) - \frac{a_n}{b_m} \cdot x^{n-m} \cdot b(x) &= a_n x^n + \cdots + a_0 - \frac{a_n}{b_m} \cdot x^{n-m} \cdot b(x) \\
&= a_n x^n +\cdots + a_0 - \frac{a_n}{b_m} x^{n-m}(b_m x^m + b_{m-1} x^{m-1} \cdots + b_0)\\
&= a_n x^n - a_n x^n +  (a_{n-1}-\frac{a_n}{b_m} b_{m-1}) x^{n-1}+  \cdots \\ &\hphantom{{}=x}+ (a_{n-m} -\frac{a_n}{b_m} b_0)x^{n-m} + a_{n-m-1}x^{n-m-1}+\cdots + a_0 \\
&= (a_{n-1}-\frac{a_n}{b_m} b_{m-1}) x^{n-1}+ \cdots + a_0
\end{flalign*}
The right hand side has degree $n-1$ therefore by the induction hypothesis there exists some $q'(x), r'(x)$ such that
\begin{flalign*}
a(x) - \frac{a_n}{b_m} \cdot x^{n-m} \cdot b(x)  &= q'(x)\cdot b(x) + r'(x) \\
&\Rightarrow a(x) = (q'(x) + \frac{a_n}{b_m} \cdot x^{n-m} ) \cdot b(x) + r'(x) 
\end{flalign*}

Thus $q(x) = (q'(x) + \frac{a_n}{b_m} \cdot x^{n-m})$ and $r(x)=r(x)'$.\\


Note that Euclidean division theorem does not state how to perform the division.
That is the topic of division algorithms.

\subsubsection{Division Algorithm}

Many division algorithms exist, the one with which most individuals
are familiar with is long division, which is commonly confused with
the concept of division itself. Other algorithms for division exist.
One of them is remarkably simple and often overlooked. For example,
if a model for multiplication is repeated addition then a model for
division could be repeated subtraction.

\paragraph{Example:}

Divide 21 by 4 using repeated subtraction. Subtract 4 from 21, $21-4=17$,
repeat with the new result $17-4=13$, repeat $13-4=9$, $9-4=5$,
$5-4=1$, stop because $4>1$. So 21 divided by 4 has a remainder
1, and the quotient is 5, because we subtracted 4 five times.

There are many other algorithms for computing the quotient and remainder
in a division problem. All of which rely on Euclidean division.

\section{Odd degree real valued polynomials with real coefficients have a
real root}

In the base case of the proof of the fundamental theorem of algebra
it was argued that odd degree polynomials with real coefficients have
a real root. A rigorous proof is provided here. First we offer some
preliminaries: we define what it means to be continuous as well as
the property that a non-empty set with an upper bound must have a
least upper bound is given. From here we then prove that polynomials
are continuous. Then by invoking the previously stated property we
can show that the least upper bound is the root of the polynomial.

\subsection{Preliminaries}

\subsubsection{Definition of a real valued function which is continuous at a point}

A real valued function, $f$ is continuous at some real number $c$
if the limiting value of $f(x)$ as $x$ approaches $c$ is equal
to $f(c)$ \cite{lang2013undergraduate}, i.e., 
\[
\lim_{x\rightarrow c}f(x)=f(c).
\]


\subsubsection{Bounding values}

An upper bound of a set is an element that is greater than or equal
to every element of the set. For example in the set $\{0,1,2,3,4\}$
an upper bound is 4, but 5 is also an upper bound. The set of upper
bounds (in the integers) for this example is the set, $U=\{4,5,6,\ldots\}$.
This is where the least upper bound comes in. In order to differentiate
between all the possible upper bounds; the least upper bound is defined
to be the smallest of all the upper bounds. If we consider the previous
example then the smallest value of $U$ is 4, thus the least upper
bound of the set $\{0,1,2,3,4\}$ is 4. \\


\subsubsection{Least upper bound property}

The least upper bound property is a statement that if certain sets
have an upper bound then they must have a least upper bound. In terms
of real values the least upper bound property is often stated as:\\

Any non-empty set of real numbers that has an upper bound must have
a least upper bound in the real numbers.

\subsection{Proof that odd degree real valued polynomials with real coefficients
have a real root}

\subsubsection*{Polynomials are continuous functions}

Here we prove that polynomials are continuous.  We will need the sum and product rule for limits: 

Let $f,g$ be real valued $\Lim{x\rightarrow c}f(x)=l$ and $\Lim{x\rightarrow c}g(x)=k$.
Then: $\Lim{x\rightarrow c}(f(x)+g(x))=l+k$.

Let $f,g$ be real valued $\Lim{x\rightarrow c}f(x)=l$ and $\Lim{x\rightarrow c}g(x)=k$.
Then: $\Lim{x\rightarrow c}(f(x)g(x))=lk$.

We will also need that the identity function $f\left(x\right)=x$
is continuous and that any constant function $f\left(x\right)=c$.

But any polynomial is just made by taking sums and products many times,
starting with the identity function and constant functions, so we
are done.

\subsubsection{Asymptotic behavior of an odd degree polynomial}

Let $P(x)=x^{d}+r_{d-1}x^{d-1}+\cdots+r_{1}x+r_{0}$, where $d$
is odd. We want to show that $P(x)$ has a real root. 
\[
x^{d}+r_{d-1}x^{d-1}+\cdots+r_{1}x+r_{0}.
\]

Recall: Since $d$ is odd,
\[
\lim_{x\rightarrow-\infty}x^{d}\rightarrow-\infty
\]
and furthermore
\[
\lim_{x\rightarrow-\infty}1+\frac{r_{d-1}}{x^{d-1}}+\cdots+\frac{r_{1}}{x}+\frac{r_{0}}{x^{d}}=1+0+\ldots+0+0=1.
\]

Thus: 
\[
\begin{array}{c}
\lim_{x\rightarrow-\infty}P(x)=\lim_{x\rightarrow-\infty}x^{d}+r_{d-1}x^{d-1}+\cdots+r_{1}x+r_{0}\\
=\lim_{x\rightarrow-\infty}x^{d}\left(1+\frac{r_{d-1}}{x^{d-1}}+\cdots+\frac{r_{1}}{x}+\frac{r_{0}}{x^{d}}\right)\\
=\lim_{x\rightarrow-\infty}x^{d}\text{�}\lim_{x\rightarrow-\infty}\left(1+\frac{r_{d-1}}{x^{d-1}}+\cdots+\frac{r_{1}}{x}+\frac{r_{0}}{x^{d}}\right)\\
=1\text{�}\left(-\infty\right)=-\infty.
\end{array}
\]
The same argument shows that $\lim_{x\rightarrow\infty}P(x)=\infty$.

Having shown that $\lim_{x\rightarrow-\infty}P(x)=-\infty$ and that
$\lim_{x\rightarrow\infty}P(x)=\infty$, we consider the set $B$
of all $x$ values, where $P(x)<0$, i.e., $B=\{x\in\mathbb{R}|P(x)<0\}$.
We know that $B$ is not the empty set, because $\lim_{x\rightarrow-\infty}P(x)=-\infty$.
Furthermore we know $B$ has an upper bound, because $\lim_{x\rightarrow\infty}P(x)=\infty$.  Since $B$ has an upper bound, it must therefore
have a least upper bound, call this value $b$.

\subsubsection{Putting it all together}

Considering again $B=\{x\in\mathbb{R}|P(x)<0\}$, for each counting number $i$, there is a number $b_{i}\in B$
such that 
\[
b_{i}>b-\frac{1}{i}
\]
since otherwise $b-\frac{1}{i}$ would be an upper bound for $B$
contradicting the fact that $b$ is the least upper bound. But this
means that, for each $i$,
\[
b-\frac{1}{i}<b_{i}\leq b
\]
\[
b=\lim_{i\rightarrow\infty}\left(b-\frac{1}{i}\right)\leq\lim_{i\rightarrow\infty}b_{i}\leq\lim_{i\rightarrow\infty}b=b.
\]
So 
\[
\lim_{i\rightarrow\infty}b_{i}=b
\]
and so, since $P\left(x\right)$ is continuous,
\[
\lim_{i\rightarrow\infty}P\left(b_{i}\right)=P\left(b\right).
\]
Furthermore, since $P\left(b_{i}\right)<0$ for all $i$,
\[
\lim_{i\rightarrow\infty}P\left(b_{i}\right)\leq0.
\]
Putting these two last facts together, we conclude that
\[
P\left(b\right)\leq0.
\]

On the other hand for each counting number $i$, there is a number
$c_{i}$ such that $b<c_{i}<b+\frac{1}{i}$ . So
\[
\lim_{i\rightarrow\infty}c_{i}=b
\]
by the same argument we used above. And so, since $P\left(x\right)$
is continuous,
\[
\lim_{i\rightarrow\infty}P\left(c_{i}\right)=P\left(b\right).
\]

But $b<c_{i}$ means that $c_{i}\notin B$ and so $P\left(c_{i}\right)\geq0$.
But this last inequality implies that
\[
\lim_{i\rightarrow\infty}P\left(c_{i}\right)\geq0.
\]
So 
\[
P\left(b\right)\geq0.
\]

But the only number that is both less than or equal to zero and greater than or equal to zero is $0$. So
\[
P\left(b\right)=0
\]
and $b$ is the root of $P\left(x\right)$ that we were looking for.

\section{Degree of $G_s(x)$}
In the proof of the fundamental theorem of algebra we showed that the degree of $G_s(x)$ is $2^{k-1}m'$ using a combinatorial argument, provided here is an alternative way to derive it.

There are $d-1$ terms that
have $i=1$, then there are $d-2$ terms that have $i=2$, because
$i<j$. So there are $d-3$ terms that have $i=3$ and so on. 
\[
G_{s}(x)=\underbrace{(x-y_{s,1,2})(x-y_{s,1,3})\cdots(x-y_{s,1,d})}_{d-1}\underbrace{(x-y_{s,2,3})\cdots(x-y_{s,2,d})}_{d-2}\cdots\underbrace{(x-y_{s,d-1,d}}_{1})
\]

So the degree of $G_{s}(x)$ is 
\[
\left(d-1\right)+\left(d-2\right)+\ldots+2+1.
\]
But doubling the series gives: 
\[
\begin{array}{c}
\left(d-1\right)+\left(d-2\right)+\ldots+2+1+\\
1+2+\ldots\left(d-2\right)+\left(d-1\right)\\
=d\cdot\left(d-1\right)
\end{array}
\]
since we doubled the series we must divide by 2, thus there are:
\[
\binom{d}{2} = \frac{d(d-1)}{2}
\]
terms in $G_s(x)$.








\section*{The Mapping $\Phi$}

Let $f(x)=a_{n}x^{n}+a_{n-1}x^{n-1}+\cdots+a_{0}$, where $a_{i}\in\mathbb{Z}$,
for $i=0,1,2,\ldots,n$.\\

Define: 
\[
\Phi_{b}(f(x))=\int f(x)\delta(b-x)dx=f(b)
\]
where $\delta$ is the dirac delta, and $b$ is the base of the integer
representation we are mapping into.

So if we were mapping into the base-10 representation of the integers,
denoted $\mathbb{Z}_{10}$, we would have: 
\[
\Phi_{10}(f(x))=\int f(x)\delta(10-x)dx=f(10)
\]
It should be stated that $\Phi$ maps the polynomials with integer
coefficients to the integers.

\subsection*{$\Phi$ preserves addition and multiplication}

%In order to show that $\Phi$ is a ring homomorphism it is necessary to show that it is closed under addition and multiplication, \emph{but I also need to show $\Phi(1_F) = 1$}.

Given two polynomials with integer coefficients $f_{1},f_{2}$.

\paragraph*{Preserves addition}

\begin{align*}
\Phi_{b}(f_{1}(x)+f_{2}(x)) & =\int(f_{1}(x)+f_{2}(x))\delta(b-x)dx\\
 & =\int f_{1}(x)\delta(b-x)dx+\int f_{2}(x)\delta(b-x)dx\\
 & =f_{1}(b)+f_{2}(b)\\
 & =\Phi_{b}(f_{1}(x))+\Phi_{b}(f_{2}(x))
\end{align*}


\paragraph{Preserves multiplication}

\begin{align*}
\Phi_{b}(f_{1}(x)f_{2}(x)) & =\int f_{1}(x)f_{2}(x)\delta(b-x)dx\\
 & =f_{1}(b)f_{2}(b)-\int f'_{1}(x)g_{2}(x)dx\text{ By integrating by parts }\\
 & =f_{1}(b)f_{2}(b)\\
 & =\Phi_{b}(f_{1}(x))\Phi_{b}(f_{2}(x))
\end{align*}

Where $\int f'_{1}(x)g_{2}(x)dx=0$ because $g_{2}(b)=f_{2}(b)$\footnote{$f_{2}(b)\in\mathbb{Z}$}
when $x=b$, but $g_{2}(x)$ is zero everywhere else, so the integral
has measure zero.
\end{appendices} 


\bibliographystyle{apacite}
\bibliography{thesis}





\end{document}
